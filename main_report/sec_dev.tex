\section{ПРОЕКТИРОВАНИЕ СТРУКТУРИРОВАННОЙ КАБЕЛЬНОЙ СИСТЕМЫ}
\label{sec:dev}

В данном разделе находится описание выбора кабелей, розеток, монтаж и размещение оборудования,
расчёт качества связи беспроводной сети для выстраиваемой ЛКС.
План этажа представлен в приложении В.
Используемые условно-графические обозначения описаны в левой части схемы.
Перечень оборудования, изделий и материалов представлен в приложении Г.

При проектировании локальной компьютерной сети значительную её часть занимает проектирование структурированной кабельной системы.
Основой проектирования структурированной кабельной системы является разводка кабелей с целью обеспечения подключений сетевого оборудования и оконечного оборудования между собой.
В данной структурированной кабельной системе для этих целей будет использоваться кабель вида витая пара.

В данном проекте кабель будет проложен в кабельном коробе вдоль стен на расстоянии 30 сантиметров от потолка,
при возникновении необходимости провести кабель сквозь стену, предполагается просверлить её и пустить через неё кабель.
Информационные розетки в кабинетах будут вмонтированы в стену на высоте 50 сантиметров от пола.
Для проводки кабеля непосредственно к информационной розетке, необходимо для начала провести кабель на предписываемом расстоянии от потолка так,
чтобы он располагался над розеткой, а затем опустить короб с кабелем перпендикулярно плоскости пола до розетки.
Прокладка кабеля между этажами осуществляется сквозь потолок в одном, обозначенном на схеме, месте.

Точки доступа расположены колличестве двух штук на нулевом этаже в геометрических центрах половин здания.
Точки монтируются к потолку, а кабели для них проводятся над фальш-потолком в коробах.

Файловый сервер, маршрутизатор, один из коммутаторов и кабельный модем располагаются на первом этаже в кабинете системного администратора,
в специальном настенном телекоммуникационном шкафу.
Телекоммуникационные шкафы предписывается монтировать на высоте 150 сантиметров от пола.

Стационарные пользовательские станции, принтер и IP-телефоны располагаются в кабинетах, в которых установлены информационные розетки.

%\subsection{Расчёт качества покрытия беспроводной сетью}
%
%Беспроводная сеть должна обеспечивать подключение 20 устройств и покрывать всю площадь помещений.
%Затухание радиоволн в беспрепятственной воздушной среде рассчитывается по упрощенной формуле:
%\\
%\[L = 32,44 + 20lg(F) + 20lg(D), dB\]
%\\
%где F – частота сигнала (ГГц), D – расстояние (м).

%\subsection{Установка точек доступа и расчет затухания сигнала}
%
%Предполагается размещение точек доступа в центре комнаты 4 и у двери комнаты 2. Все точки доступа устанавливаются на потолке.
%Для этого для их проведен кабель по потолку. Подключение питания для точек доступа не требуется, так как используется PoE.
%Если учесть, что высота этажа три метра, то самая удаленная точка в любой из комнат находится на расстоянии 9-ти метров.
%Воспользуемся упрощенной формулой для расчета затухания сигнала:
%
%\[L = 32,44 + 20lg(F) + 20lg(D), dB\]
%
%Где \(F\) – это частота в ГГц, \(D\) – расстояние до самого удалённого мобильного устройства в метрах.
%
%В качестве частоты возьмем 2,4 ГГц, так как эта частота является стандартной рабочей частотой установленной после базового,
%.
%Рассчитаем затухание волны:
%
%\[L = 32,44 + 20lg(2,4) + 20lg(9) = 59,13 dB\]
%
%Мощность передатчика точки доступа составляет 20 дБм, для того чтобы узнать какой уровень сигнала будет на таком
%расстоянии следует от мощности передатчика отнять вычисленное выше затухание сигнала, итого уровень сигнала на
%таком расстоянии равен -39,13 дБм, что соответствует хорошему сигналу, в самой удаленной точке помещения.
%
%Также, при проектировании плана этажа учитывалось то, что точки доступа будут использоваться в помещениях, где работают сотрудники
%и при переходе из одного помещения в другое.
