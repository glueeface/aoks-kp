\sectionCenteredToc{ВВЕДЕНИЕ}
\label{sec:intro}

%\referenceTitle

Целью курсового проекта является проектирование локальной компьютерной сети для организации, которая занимается торговлей компьютерными комплектующими.

Задачами при реализации курсового проекта являются:
разработка структуры сети,
структурной схемы;
выбор устройств,
обоснование их выбора,
описание настройки устройств,
составление функциональной схемы,
подведение итогов.

В контексте организации, которая занимается торговлей компьютерными комплектующими, разработка локальной компьютерной сети является первостепенной задачей, как и для любой организации современного мира, так как на её основе реализуются, например, системы охраны или безопасности.

Суммарная площадь, занимаемая помещениями организации, составляет 210 квадратных метров.
Организация располагается в здании с вытянутой прямоугольной формой на двух этажах: подвальном и первом.
Также известно количество мобильных подключений, установленное заказчиком: 20.
С подключением стационарных подключений и подключений пользовательских компьютеров не определено, что является минусом при разработке локальной компьютерной сети.
Однако их количество будет равно 10, так как характеристика сети, основанная на финансах, установленных заказчиком, – бюджетная сеть.
Из числа прочих оконечных устройств заказчик указал установку и подключение принтеров и IP-телефонов.
Дополнительно должен быть организован доступ к NTFS файловому серверу внутри ЛКС.
Подключение к Internet будет осуществляться по агрегированному каналу при помощи витой пары Gigabit Ethernet.

Должна быть реализована защита локальной компьютерной сети от вирусов,
а также повышенная пожарная безопасность, исходя из требований заказчика.
Также заказчик указал конкретного производителя оборудования – Allied Telesis.
Это международная компания, специализирующаяся на телекоммуникации.
Это условие значительно облегчает выбор аппаратуры при проектировании ЛКС.

\newpage

