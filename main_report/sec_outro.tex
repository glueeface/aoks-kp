\sectionCenteredToc{ЗАКЛЮЧЕНИЕ}
\label{sec:outro}

В ходе выполнения курсового проекта были применены на практике теоретические и практические знания, такие как
выбор оборудования, настройка и его размещение.
Были выполнены все поставленные требования, а именно:

\begin{itemize}
  \item Учтена специфика объекта;
  \item Рассчитаны количество стационарных пользователей и количество стационарных подключений;
  \item Установлены принтеры и IP-телефоны;
  \item Организовано подключение к Интернету через агрегированный канал
  \item Адресация выполнена согласно заданию;
  \item Установлена и сконфигурирован файловый сервер;
\end{itemize}

Помимо требований задания, для организации качественной ЛКС была выполнена
организация сети для мобильных подключений и было выбрано сетевое оборудование
компании Allied Telesis.

Среди достоинств данной сети можно выделить:

\begin{itemize}
  \item Защищенность сети;
  \item Безопасность персонала и имущества;
  \item Повышенная пожарная безопасность;
  \item Возможность расширения в будущем;
\end{itemize}

Недостатками построенной ЛКС являются:

\begin{itemize}
  \item Низкая производительность моноблоков;
  \item Использование файлового сервера с одним слотом памяти;
\end{itemize}

В данной сети в будущем можно будет увеличить уровень безопасности включением IPsec и VPN.
Также в данную сеть можно будет добавить ещё один файловый сервер
для увеличения производительности и объёма хранимых данных,
сменить ПК или поставить дополнительные для отделов, которым понадобиться высокая вычислительная нагрузка.

\newpage

ujdyj
