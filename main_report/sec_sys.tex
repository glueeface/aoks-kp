\section{СТРУКТУРНОЕ ПРОЕКТИРОВАНИЕ}
\label{sec:sys}

В данном разделе описана структура локальной компьютерной сети.
Структурное проектирование и разработка структурной схемы нужны для упрощения разработки локальной компьютерной сети.
Структурная схема содержит в себе логические блоки, на которые разделена локальная компьютерная сеть.
Исходя из логических блоков можно получить представление о разрабатываемой локальной компьютерной сети, не вникая в реализацию программно-аппаратных средств.
Логические блоки, представленные на структурной схеме следующие:
\begin{itemize}
    \item блок выхода в интернет
    \item блок \moduleRouting
    \item блок \moduleCommutation
    \item блок \moduleEndDevices
    \item блок \moduleWiFi
    \item блок мобильных \moduleEndDevices
    \item блок \moduleNTFS
\end{itemize}

Всего предполагается организовать 20 беспроводных подключений.
С количеством стационарных подключений и числом персональных компьютеров заказчик не определился
и эти значения будут рассчитаны на основании бюджета, предоставленного заказчиком,
а также на анализе локальных компьютерных сетей схожих организаций.
Также заказчик указал в требованиях подключение дополнительных оконечных устройств: принтеров и IP-телефонов.

Взаимосвязь между основными компонентами проекта отражена в приложении А.

\subsection{Блок \moduleRouting}\label{subsec:struct:ModuleRouting}

Блок \moduleRouting\ является ключевым элементом в сетевой инфраструктуре, ответственным за определение оптимального пути передачи данных между различными сегментами сети.
Этот блок вынесен отдельно, так как реализует следующий функционал:
\begin{itemize}
    \item определение оптимального маршрута для передачи данных,
    \item обеспечение сегментации сети,
    \item управление полосой пропускания,
    \item обеспечение безопасности и фильтрации трафика,
    \item поддерживание взаимодействия между сегментами сети.
%    \item маршрутизация внутри локальной компьютерной сети;
%    \item доступ к Интернету для устройств, подключённых к локальной компьютерной сети.
\end{itemize}

%Блок \moduleRouting\ представлен маршрутизатором и связан с блоками коммутации.
%Он определяет, каким образом данные будут передаваться между устройствами внутри локальной сети (LAN) и внешними сетями (WAN, Интернет).
%Взаимодействие с блоками коммутации необходимо для эффективного маршрутизации внутри локальной сети.
С одной стороны, маршрутизаторы этого блока подключены к провайдеру интернет-услуг, что позволяет организации при помощи оконечных устройств иметь доступ к глобальной сети.
С другой стороны, они связаны с блоком коммутации, что обеспечивает передачу данных между всеми устройствами в локальной сети.
Таким образом, блок маршрутизации является центральным элементом сети, обеспечивая ее связность, безопасность и доступность, что позволяет всем устройствам в организации успешно взаимодействовать как между собой, так и с внешним миром.

\subsection{Блок \moduleCommutation}\label{subsec:struct:ModuleCommutation}
    Блок коммутации является связующим звеном в структуре локальной компьютерной сети организации.
Это видно на структурной схеме – он связывает блок маршрутизации с остальными блоками локальной компьютерной сети.
Блок коммутации состоит из нескольких коммутаторов, объединенных в иерархию, которая обеспечивает эффективное управление и передачу данных внутри сети.
Общее взаимодействие блока коммутации с другими блоками включает в себя передачу данных от маршрутизаторов через коммутаторы к конечным устройствам, а также обеспечение беспроводного доступа к сети через точки доступа. \\
    Первый из блоков, с которым взаимодействует блок коммутации, является блок оконечных устройств.
Каждый коммутатор в этом блоке подключен к группе стационарных ПК и принтеров, IP-телефонов, представленных блоком оконечных устройств.
Коммутаторы обеспечивают связи для передачи данных между ПК и принтерами, между IP-телефонами, подключёнными к локальной компьютерной сети организации.
Это позволяет сотрудникам обмениваться информацией и отправлять на печать документы.
    Следующий блок, взаимодействующий с блоком коммутации, является блок точек доступа.
Данный блок предоставляет беспроводное подключение для устройств, таких как ноутбуки и мобильные устройства.
Коммутаторы взаимодействуют с точками доступа, обеспечивая беспроводную связь для сотрудников, что позволяет им работать в сети без физического подключения к коммутаторам.

\subsection{Блок \moduleEndDevices}\label{subsec:struct:EndDevices}

Оконечные устройства представляют собой компьютеры, ноутбуки, принтеры и другие устройства, которые являются конечными точками для пользователей в сети.
Эти устройства подключаются к сети для обмена данными и доступа к различным ресурсам, таким как файловые серверы, принтеры и Интернет.

\subsection{Блок \moduleWiFi}\label{subsec:struct:WiFi}

Согласно требованию заказчика, должно быть предусмотрено подключение беспроводных устройств,
для реализации данного требования, с учётом площади помещения, необходима установка нескольких беспроводных точек доступа на каждом этаже.
Подключать беспроводные точки доступа напрямую к маршрутизатору нет необходимости, поэтому точки доступа соединены с маршрутизатором посредством коммутаторов.

\subsection{Блок мобильных \moduleEndDevices}\label{subsec:struct:WirelessEndDevices}

Блок мобильных оконечных устройств вынесен отдельным блоком, так как подключение к локальной компьютерной сети будет производиться не посредством кабеля,
а при помощи беспроводного соединения.
К блоку мобильных оконечных устройств относятся, например, мобильные телефоны работников и посетителей организации.

\subsection{Блок \moduleNTFS}\label{subsec:struct:Server}

Файловый сервер предоставляет централизованное хранилище данных для сети.
Этот сервер обеспечивает управление файлами, обмен данными и обеспечивает доступ к общим ресурсам.
Пользователи могут сохранять и извлекать файлы с файлового сервера, что способствует централизации данных и их обеспечению безопасностью и резервным копированием.

NTFS/SMB-сервер не будет подключён напрямую к маршрутизатору.
Такое решение обусловлено тем, что файловый сервер будет использоваться всеми участниками внутри сети,
однако доступа к нему из сети Интернет быть не должно.
